%%%%%%%%%%%%%%%%%%%%%%%%%%%%%%%%%%%%%%%%%
% baposter Landscape Poster
% LaTeX Template
% Version 1.0 (11/06/13)
%
% baposter Class Created by:
% Brian Amberg (baposter@brian-amberg.de)
%
% This template has been downloaded from:
% http://www.LaTeXTemplates.com
%
% License:
% CC BY-NC-SA 3.0 (http://creativecommons.org/licenses/by-nc-sa/3.0/)
%
%%%%%%%%%%%%%%%%%%%%%%%%%%%%%%%%%%%%%%%%%

%----------------------------------------------------------------------------------------
%	PACKAGES AND OTHER DOCUMENT CONFIGURATIONS
%----------------------------------------------------------------------------------------

\documentclass[landscape,a0paper,fontscale=0.285]{baposter} % Adjust the font scale/size here

\usepackage{graphicx} % Required for including images
\graphicspath{{figures/}} % Directory in which figures are stored

\usepackage{amsmath} % For typesetting math
\usepackage{amssymb} % Adds new symbols to be used in math mode

\usepackage{booktabs} % Top and bottom rules for tables
\usepackage{enumitem} % Used to reduce itemize/enumerate spacing
\usepackage{palatino} % Use the Palatino font
\usepackage[font=Large,labelfont=bf]{caption} % Required for specifying captions to tables and figures

\usepackage{multicol} % Required for multiple columns
\setlength{\columnsep}{0.4em} % Slightly increase the space between columns
\setlength{\columnseprule}{0mm} % No horizontal rule between columns

\usepackage{tikz} % Required for flow chart
\usepackage{onimage}
\usetikzlibrary{shapes,arrows} % Tikz libraries required for the flow chart in the template

\newcommand{\compresslist}{ % Define a command to reduce spacing within itemize/enumerate environments, this is used right after \begin{itemize} or \begin{enumerate}
\setlength{\itemsep}{1pt}
\setlength{\parskip}{0pt}
\setlength{\parsep}{0pt}
}

\definecolor{lightblue}{rgb}{0.145,0.6666,1} % Defines the color used for content box headers

\begin{document}

\begin{poster}
{
headerborder=closed, % Adds a border around the header of content boxes
colspacing=1em, % Column spacing
bgColorOne=white, % Background color for the gradient on the left side of the poster
bgColorTwo=white, % Background color for the gradient on the right side of the poster
borderColor=lightblue, % Border color
headerColorOne=black, % Background color for the header in the content boxes (left side)
headerColorTwo=lightblue, % Background color for the header in the content boxes (right side)
headerFontColor=white, % Text color for the header text in the content boxes
boxColorOne=white, % Background color of the content boxes
textborder=roundedleft, % Format of the border around content boxes, can be: none, bars, coils, triangles, rectangle, rounded, roundedsmall, roundedright or faded
eyecatcher=true, % Set to false for ignoring the left logo in the title and move the title left
headerheight=0.1\textheight, % Height of the header
headershape=roundedright, % Specify the rounded corner in the content box headers, can be: rectangle, small-rounded, roundedright, roundedleft or rounded
headerfont=\LARGE\bf\textsc, % Large, bold and sans serif font in the headers of content boxes
%textfont={\setlength{\parindent}{1.5em}}, % Uncomment for paragraph indentation
linewidth=2pt % Width of the border lines around content boxes
}
%----------------------------------------------------------------------------------------
%	TITLE SECTION 
%----------------------------------------------------------------------------------------
%
{\includegraphics[height=5em]{KICP.png}\includegraphics[scale=0.28,trim=10 5 10 10,clip]{Yale.jpg}} % First university/lab logo on the left
%{\includegraphics[scale=0.3]{Yale.png}} % First university/lab logo on the left
{\bf\textsc{Stirred, not Clumped: the Evolution of Temperature Profiles in the Outskirts of Galaxy Clusters}\vspace{0.5em}} % Poster title
{\textsc{ {\Large Camille Avestruz$^{1}$ , Daisuke Nagai$^2$, Erwin T.H. Lau$^2$}  \hspace{10pt} {\large $^1$University of Chicago, $^2$ Yale University}}} % Author names and institution
{\includegraphics[width=5.5em, trim=60 155 15 10, clip]{HeadShot.jpg}} % Second university/lab logo on the right

%----------------------------------------------------------------------------------------
%	ABSTRACT
%----------------------------------------------------------------------------------------

\headerbox{Summary}{name=abstract,column=0,span=2,row=0}{ \Large
  X-ray measurements of the intracluster medium (ICM) suggest that
  temperature profiles at large radii deviate from self-similar
  evolution. Using a mass-limited sample of galaxy clusters from
  cosmological hydrodynamical simulations, we show that {\bf the departure
  from self-similarity is due to physical processes that are driven by
  mergers and accretion. The dominant contribution comes from the
  evolution of non-thermal gas motions that have not yet thermalized
  (stirring).} The evolution of accreting cool, dense gas substructures
  that bias the average temperature low (clumping), is subdominant. We
  can mitigate departures from self-similarity with a careful choice
  of halo overdensity definition that scales out the accretion
  dependence. These results highlight the importance of understanding
  non-thermal gas motions in the ICM and the use of galaxy clusters as
  cosmological probes.
\vspace{0.3em} % When there are two boxes, some whitespace may need to be added if the one on the right has more content
}

%----------------------------------------------------------------------------------------
%	INTRODUCTION
%----------------------------------------------------------------------------------------

%% \headerbox{Introduction}{name=introduction,column=1,row=0,bottomaligned=objectives}{

%% Aliquam non lacus dolor, \textit{a aliquam quam}. Cum sociis natoque penatibus et magnis dis parturient montes, nascetur ridiculus mus. Nulla in nibh mauris. Donec vel ligula nisi, a lacinia arcu. Sed mi dui, malesuada vel consectetur et, egestas porta nisi. Sed eleifend pharetra dolor, et dapibus est vulputate eu. \textbf{Integer faucibus elementum felis vitae fringilla.} In hac habitasse platea dictumst. Duis tristique rutrum nisl, nec vulputate elit porta ut. Donec sodales sollicitudin turpis sed convallis. Etiam mauris ligula, blandit adipiscing condimentum eu, dapibus pellentesque risus.
%% }

%----------------------------------------------------------------------------------------
%	Cluster Outskirts and Accretion
%----------------------------------------------------------------------------------------

\headerbox{Accretion stirs and clumps the outskirts}{name=accretion,column=0,span=2,below=abstract}{
\vspace{-1em}
\begin{multicols}{2}
\begin{center}\begin{tikzonimage}[width=\columnwidth, trim=0 120 500 600, clip]{figures/Abell85Composite.png}
      %\node at ([text,text width=3cm, align=center] 0.52,0.53) {\small\bf\color{white} R2500c};
      %%%%%%%%%%%%%%%%%%%%%%                                                                                                                         
        \draw [cyan, line width=1.5pt] (0.49,0.55) circle [radius=1.2cm];
        \node at ([text, cyan,text width=3cm, align=center] 0.49,0.76) {\normalsize\bf\color{cyan} R500c};
      %%%%%%%%%%%%%%%%%%%%%%                                                                                                                         
        \draw [cyan, line width=1.5pt] (0.49,0.55) circle [radius=1.8cm];
        \node at ([text, cyan,text width=3cm, align=center] 0.49,0.86) {\normalsize\bf\color{cyan} R200c};
      %%%%%%%%%%%%%%%%%%%%%%                                                                                                                         
        \draw [white, line width=1.5pt] (0.49,0.55) circle [radius=3.6cm];
        \node at ([text, white,text width=3cm, align=center] 0.78,.85) {\normalsize\bf\color{white} R200m};

        \node at (0.15,0.2) {\includegraphics[scale=.1, trim=0 0 0 20, clip]{figures/perseusCluster_cxc_f.jpg}};
        \draw [lime, line width=1.5pt] (0.08,0.39) -- (0.42,0.55);
        \draw [lime, line width=1.5pt] (0.32,0.04) -- (0.5,0.45);
        \draw [lime, line width=1.5pt] (0.49,0.55) circle [radius=0.6cm];
        \node at ([text,text width=3cm, align=center] 0.5,0.67) {\normalsize\bf\color{lime} R2500c};
        \node at ([text,text width=3cm, align=center] 0.16,0.34) {\normalsize\bf\color{lime} Heating/Cooling};
        \node at ([text,text width=3cm, align=center] 0.15,0.28) {\normalsize\bf\color{lime} dependent};
        \node at ([text,text width=3cm, align=center] 0.15,0.08) {\large\bf\color{lime} Cluster Core};
      %%%%%%%%%%%%%%%%%%%%%%                                                                                                                         
      \draw [<->,line width=3pt,pink] (0.55,0.5) -- (0.58,0.41);
      \node at ([text, pink,text width=3cm, align=center] 0.62,0.36) {\large\bf\color{pink} Least};
      \node at ([text, pink,text width=3cm, align=center] 0.66,0.29) {\large\bf\color{pink} Affected};
      %%%%%%%%%%%%%%%%%%%%%%                                                                                                                         
      \draw [->,line width=3pt,orange] (0.65,0.53) -- (0.94,0.53);
      \node at ([text, white,text width=3cm, align=center] 0.8,0.72) {\large\bf\color{orange} Accretion};
      \node at ([text, white,text width=3cm, align=center] 0.8,0.66) {\large\bf\color{orange} dependent};
      \node at ([text, white,text width=3cm, align=center] 0.805,0.58) {\Large\bf\color{orange} Outskirts};

    \end{tikzonimage}
%\captionof{figure}{Most observations ...}
\end{center}
\begin{center}
\begin{itemize}
\alignleft
\item[]
\item {\Large Accretion generates non-thermal gas motions, and carry
  in denser cooler gas with subhalos and penetrating filaments.}
\item {\Large On the other hand, cooling, star formation, and feedback
  processes drive ICM behavior in cluster cores.}
\item {\Large Gas between $R_{2500c}$ and $R_{500c}$ is the least
  affected by astrophysics.}

%\item {\Large Galaxy clusters experience more mergers and accretion
%  earlier in their history.}
\end{itemize}
\end{center}
\end{multicols}
}


%----------------------------------------------------------------------------------------
%	Key Question
%----------------------------------------------------------------------------------------

\headerbox{High redshift clusters accrete more}{name=question,column=2,span=2}%bottomaligned=accretion}{ % This block's bottom aligns with the bottom of the conclusion block
{
\begin{multicols}{2}
\vspace{1em}
\begin{center}
\includegraphics[width=\linewidth,trim=30 190 50 -20,
  clip]{figures/SuperClumpingCartoon_lowzclumps.png}
\captionof{figure}{\Large\alignleft Cartoon of high redshift clusters with
  relatively more accreting subhalos than low redshift
  counterparts. High redshift clusters have relatively more
  newly accreted material, and recently stirred gas motions to thermalize.}
\end{center}
%----------------------------------------------------%
\begin{center}
    \begin{tikzonimage}[width=\linewidth,trim=18 18 18 18, clip]{figures/f4.pdf}
      %\node at ([text, text width=3cm, align=center] 0.83,.07) {\color{black}\bf\small Avestruz+16};
      \draw [->,line width=2pt,red] (0.61,0.43) -- (0.61,.48);
      \draw [->,line width=2pt,red] (0.61,0.11) -- (0.61,.17);
      \node at ([text, text width=3cm, align=center] 0.55,.33) {\color{red}\bf\large $\sim30\%$ increase};
      \node at ([text, text width=3cm, align=center] 0.55,.28) {\color{red}\bf\large in non-bulk};
      \node at ([text, text width=3cm, align=center] 0.55,.23) {\color{red}\bf\large contributions};
      \end{tikzonimage}
\captionof{figure}{\Large\alignleft (Right) Density profile ratio of
  all gas to ``bulk'' gas, {\it not} in accreting
  substructure.\cite{avestruz_etal16}}
\end{center}


\end{multicols}
}
%----------------------------------------------------------------------------------------
%	FUTURE RESEARCH
%----------------------------------------------------------------------------------------

%% \headerbox{Future Research}{name=futureresearch,column=1,span=2,aligned=references,above=bottom}{ % This block is as tall as the references block

%% \begin{multicols}{2}
%% Integer sed lectus vel mauris euismod suscipit. Praesent a est a est ultricies pellentesque. Donec tincidunt, nunc in feugiat varius, lectus lectus auctor lorem, egestas molestie risus erat ut nibh.

%% Maecenas viverra ligula a risus blandit vel tincidunt est adipiscing. Suspendisse mollis iaculis sem, in \emph{imperdiet} orci porta vitae. Quisque id dui sed ante sollicitudin sagittis.
%% \end{multicols}
%% }

%----------------------------------------------------------------------------------------
%	CONTACT INFORMATION
%----------------------------------------------------------------------------------------

%% \headerbox{Contact Information}{name=contact,column=3,above=bottom}{ % This block is as tall as the references block

%% \begin{description}\compresslist
%% \item[Web] www.university.edu/smithlab
%% \item[Email] john@smith.com
%% \item[Phone] +1 (000) 111 1111
%% \end{description}
%% }

%----------------------------------------------------------------------------------------
%	R200m
%----------------------------------------------------------------------------------------

\headerbox{Scale out accretion}{name=R200m,column=3,span=1,below=question,above=bottom,bottomaligned=references}{
}
%----------------------------------------------------------------------------------------
%	Stirring
%----------------------------------------------------------------------------------------

\headerbox{Stirred evolution}{name=stirring,column=2,span=1,below=question,above=bottom,bottomaligned=references}{

}

%----------------------------------------------------------------------------------------
%	REFERENCES
%----------------------------------------------------------------------------------------

\headerbox{References}{name=references,column=0,span=2,below=accretion,above=bottom,bottomaligned=R200m}{

\renewcommand{\section}[2]{\vskip 0.05em} % Get rid of the default "References" section title
\nocite{*} % Insert publications even if they are not cited in the poster
\small{ % Reduce the font size in this block
\bibliographystyle{unsrt}
\bibliography{sample} % Use sample.bib as the bibliography file
}}


%----------------------------------------------------------------------------------------
%	RESULTS 2
%----------------------------------------------------------------------------------------

%% \headerbox{Results 2}{name=results2,column=1,below=objectives,bottomaligned=conclusion}{ % This block's bottom aligns with the bottom of the conclusion block

%% Donec faucibus purus at tortor egestas eu fermentum dolor facilisis. Maecenas tempor dui eu neque fringilla rutrum. Mauris \emph{lobortis} nisl accumsan.

%% \begin{center}
%% \begin{tabular}{l l l}
%% \toprule
%% \textbf{Treatments} & \textbf{Response 1} & \textbf{Response 2}\\
%% \midrule
%% Treatment 1 & 0.0003262 & 0.562 \\
%% Treatment 2 & 0.0015681 & 0.910 \\
%% Treatment 3 & 0.0009271 & 0.296 \\
%% \bottomrule
%% \end{tabular}
%% \captionof{table}{Table caption}
%% \end{center}

%% Nulla ut porttitor enim. Suspendisse venenatis dui eget eros gravida tempor. Mauris feugiat elit et augue placerat ultrices. Morbi accumsan enim nec tortor consectetur non commodo.

%% \begin{center}
%% \begin{tabular}{l l l}
%% \toprule
%% \textbf{Treatments} & \textbf{Response 1} & \textbf{Response 2}\\
%% \midrule
%% Treatment 1 & 0.0003262 & 0.562 \\
%% Treatment 2 & 0.0015681 & 0.910 \\
%% Treatment 3 & 0.0009271 & 0.296 \\
%% \bottomrule
%% \end{tabular}
%% \captionof{table}{Table caption}
%% \end{center}
%% }

%----------------------------------------------------------------------------------------

\end{poster}

\end{document}
