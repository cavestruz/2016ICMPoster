%%%%%%%%%%%%%%%%%%%%%%%%%%%%%%%%%%%%%%%%%
% baposter Landscape Poster
% LaTeX Template
% Version 1.0 (11/06/13)
%
% baposter Class Created by:
% Brian Amberg (baposter@brian-amberg.de)
%
% This template has been downloaded from:
% http://www.LaTeXTemplates.com
%
% License:
% CC BY-NC-SA 3.0 (http://creativecommons.org/licenses/by-nc-sa/3.0/)
%
%%%%%%%%%%%%%%%%%%%%%%%%%%%%%%%%%%%%%%%%%

%----------------------------------------------------------------------------------------
%	PACKAGES AND OTHER DOCUMENT CONFIGURATIONS
%----------------------------------------------------------------------------------------

\documentclass[landscape,a0paper,fontscale=0.285]{baposter} % Adjust the font scale/size here

\usepackage{graphicx} % Required for including images
\graphicspath{{figures/}} % Directory in which figures are stored

\usepackage{amsmath} % For typesetting math
\usepackage{amssymb} % Adds new symbols to be used in math mode

\usepackage{booktabs} % Top and bottom rules for tables
\usepackage{enumitem} % Used to reduce itemize/enumerate spacing
\usepackage{palatino} % Use the Palatino font
\usepackage[font=large,labelfont=bf]{caption} % Required for specifying captions to tables and figures

\usepackage{multicol} % Required for multiple columns
\setlength{\columnsep}{0.4em} % Slightly increase the space between columns
\setlength{\columnseprule}{0mm} % No horizontal rule between columns
\usepackage{tikz} % Required for flow chart
\usepackage{onimage}
\usetikzlibrary{shapes,arrows} % Tikz libraries required for the flow chart in the template

\newcommand{\compresslist}{ % Define a command to reduce spacing within itemize/enumerate environments, this is used right after \begin{itemize} or \begin{enumerate}
\setlength{\itemsep}{1pt}
\setlength{\parskip}{0pt}
\setlength{\parsep}{0pt}
}

\definecolor{lightblue}{rgb}{0.145,0.6666,1} % Defines the color used for content box headers

\begin{document}

\begin{poster}
{
headerborder=closed, % Adds a border around the header of content boxes
colspacing=1em, % Column spacing
bgColorOne=white, % Background color for the gradient on the left side of the poster
bgColorTwo=white, % Background color for the gradient on the right side of the poster
borderColor=lightblue, % Border color
headerColorOne=black, % Background color for the header in the content boxes (left side)
headerColorTwo=lightblue, % Background color for the header in the content boxes (right side)
headerFontColor=white, % Text color for the header text in the content boxes
boxColorOne=white, % Background color of the content boxes
textborder=roundedleft, % Format of the border around content boxes, can be: none, bars, coils, triangles, rectangle, rounded, roundedsmall, roundedright or faded
eyecatcher=true, % Set to false for ignoring the left logo in the title and move the title left
headerheight=0.1\textheight, % Height of the header
headershape=roundedright, % Specify the rounded corner in the content box headers, can be: rectangle, small-rounded, roundedright, roundedleft or rounded
headerfont=\LARGE\bf\textsc, % Large, bold and sans serif font in the headers of content boxes
%textfont={\setlength{\parindent}{1.5em}}, % Uncomment for paragraph indentation
linewidth=2pt % Width of the border lines around content boxes
}
%----------------------------------------------------------------------------------------
%	TITLE SECTION 
%----------------------------------------------------------------------------------------
%
{\includegraphics[height=5em]{KICP.png}\includegraphics[scale=0.28,trim=10 5 10 10,clip]{Yale.jpg}} % First university/lab logo on the left
%{\includegraphics[scale=0.3]{Yale.png}} % First university/lab logo on the left
{\bf\textsc{Stirred, not Clumped: the Evolution of Temperature Profiles in the Outskirts of Galaxy Clusters}\vspace{0.5em}} % Poster title
{\textsc{ {\Large Camille Avestruz$^{1}$ , Daisuke Nagai$^2$, Erwin T.H. Lau$^2$}  \hspace{10pt} {\large $^1$University of Chicago, $^2$ Yale University}}} % Author names and institution
{\includegraphics[width=5.5em, trim=60 155 15 10, clip]{HeadShot.jpg}} % Second university/lab logo on the right

%----------------------------------------------------------------------------------------
%	ABSTRACT
%----------------------------------------------------------------------------------------

\headerbox{Summary}{name=abstract,column=0,span=2,row=0}{ \large
  Accretion driven processes lead to an evolution in cluster outskirt
  temperature profiles scaled with respect to $R_{500c}$, breaking the
  self-similar evolution found at intermediate radii. With
  hydrodynamical simulations, we show that the dominant contribution
  comes from the evolution of non-thermal gas motions (stirring). The
  evolution of accreting cool, dense gas substructures that bias the
  average temperature low (clumping), is subdominant. The choice of
  $R_{200m}$ in halo overdensity definition scales out the accretion
  dependence.
\vspace{0.3em} % When there are two boxes, some whitespace may need to be added if the one on the right has more content
}


%----------------------------------------------------------------------------------------
%	Cluster Outskirts and Accretion
%----------------------------------------------------------------------------------------

\headerbox{Accretion stirs and clumps the outskirts}{name=accretion,column=0,span=2,below=abstract}{
\vspace{-2em}
\begin{multicols}{2}
\begin{center}
\begin{tikzonimage}[width=\columnwidth, trim=0 120 500 600, clip]{figures/Abell85Composite.png}
      %\node at ([text,text width=3cm, align=center] 0.52,0.53) {\small\bf\color{white} R2500c};
      %%%%%%%%%%%%%%%%%%%%%%                                                             
        \draw [cyan, line width=1.5pt] (0.49,0.55) circle [radius=1.2cm];
        \node at ([text, cyan,text width=3cm, align=center] 0.49,0.76) {\normalsize\bf\color{cyan} R500c};
      %%%%%%%%%%%%%%%%%%%%%%                                                                                                                         
        \draw [cyan, line width=1.5pt] (0.49,0.55) circle [radius=1.8cm];
        \node at ([text, cyan,text width=3cm, align=center] 0.49,0.86) {\normalsize\bf\color{cyan} R200c};
      %%%%%%%%%%%%%%%%%%%%%%                                                                                                                         
        \draw [white, line width=1.5pt] (0.49,0.55) circle [radius=3.6cm];
        \node at ([text, white,text width=3cm, align=center] 0.78,.85) {\normalsize\bf\color{white} R200m};

        \node at (0.15,0.2) {\includegraphics[scale=.1, trim=0 0 0 20, clip]{figures/perseusCluster_cxc_f.jpg}};
        \draw [lime, line width=1.5pt] (0.08,0.39) -- (0.42,0.55);
        \draw [lime, line width=1.5pt] (0.32,0.04) -- (0.5,0.45);
        \draw [lime, line width=1.5pt] (0.49,0.55) circle [radius=0.6cm];
        \node at ([text,text width=3cm, align=center] 0.5,0.67) {\normalsize\bf\color{lime} R2500c};
        \node at ([text,text width=3cm, align=center] 0.16,0.34) {\normalsize\bf\color{lime} Heating/Cooling};
        \node at ([text,text width=3cm, align=center] 0.15,0.28) {\normalsize\bf\color{lime} dependent};
        \node at ([text,text width=3cm, align=center] 0.15,0.08) {\large\bf\color{lime} Cluster Core};
      %%%%%%%%%%%%%%%%%%%%%%                                                                                                                         
      \draw [<->,line width=3pt,pink] (0.55,0.5) -- (0.58,0.41);
      \node at ([text, pink,text width=3cm, align=center] 0.62,0.36) {\large\bf\color{pink} Least};
      \node at ([text, pink,text width=3cm, align=center] 0.66,0.29) {\large\bf\color{pink} Affected};
      %%%%%%%%%%%%%%%%%%%%%%                                                             
      \draw [->,line width=3pt,orange] (0.65,0.53) -- (0.94,0.53);
      \node at ([text, white,text width=3cm, align=center] 0.8,0.72) {\large\bf\color{orange} Accretion};
      \node at ([text, white,text width=3cm, align=center] 0.8,0.66) {\large\bf\color{orange} dependent};
      \node at ([text, white,text width=3cm, align=center] 0.805,0.58) {\Large\bf\color{orange} Outskirts};
    \end{tikzonimage}
\end{center}
\begin{center}
\begin{itemize}
\alignleft
\item[]
\item {\large Accretion generates non-thermal gas motions, and carry
  in denser cooler gas with subhalos and penetrating filaments.}
\item {\large Gas motions provide non-thermal pressure support.
  Cooler gas can bias the measured temperature low.}
\item {\large On the other hand, cooling, star formation, and feedback
  processes drive ICM behavior in cluster cores.}
\item {\large Astrophysics least affects gas between $R_{2500c}$ and $R_{500c}$.}
%\item {\Large Galaxy clusters experience more mergers and accretion
%  earlier in their history.}
\end{itemize}
\end{center}
\end{multicols}
}


%----------------------------------------------------------------------------------------
%	Accretion evolution
%----------------------------------------------------------------------------------------

\headerbox{High redshift clusters accrete more}{name=evolvingaccretion,column=0,span=2,below=accretion,above=bottom}{ % This block's bottom aligns with the bottom of the conclusion block
\begin{multicols}{2}
\vspace{1em}
\begin{center}
\includegraphics[width=\linewidth,trim=30 190 50 20,
  clip]{figures/SuperClumpingCartoon_lowzclumps.png}
\captionof{figure}{\large\alignleft Cartoon: On average, high redshift
  clusters have relatively more newly accreted material, and recently
  stirred gas motions yet to thermalize in the ``bulk'' (non-substructure) component of the ICM.}
\end{center}
%----------------------------------------------------%
\begin{center}
    \begin{tikzonimage}[width=.9\linewidth,trim=18 18 16 17, clip]{figures/f4.pdf}
      %\node at ([text, text width=3cm, align=center] 0.83,.07) {\color{black}\bf\small Avestruz+16};
      \draw [->,line width=2pt,red] (0.61,0.43) -- (0.61,.48);
      \draw [->,line width=2pt,red] (0.61,0.11) -- (0.61,.17);
      \node at ([text, text width=3cm, align=center] 0.45,.33) {\color{red}\bf\large $\sim30\%$ increase};
      \node at ([text, text width=3cm, align=center] 0.45,.28) {\color{red}\bf\large in substructure};
      \node at ([text, text width=3cm, align=center] 0.45,.23) {\color{red}\bf\large contributions};
      \node at ([text, text width=3cm, align=center] 0.45,.18) {\color{red}\bf\large to gas density};
      \end{tikzonimage}
%% \captionof{figure}{\large\alignleft Density profile ratio of ``all''
%%   to ``bulk'' gas ({\it not} in substructure) of our simulated
%%   sample. \cite{avestruz_etal16}}
\end{center}
\end{multicols}
}
%----------------------------------------------------------------------------------------
%	CONTACT INFORMATION
%----------------------------------------------------------------------------------------

%% \headerbox{Contact Information}{name=contact,column=3,above=bottom}{ % This block is as tall as the references block

%% \begin{description}\compresslist
%% \item[Web] www.university.edu/smithlab
%% \item[Email] john@smith.com
%% \item[Phone] +1 (000) 111 1111
%% \end{description}
%% }

%----------------------------------------------------------------------------------------
%	Stirring
%----------------------------------------------------------------------------------------

\headerbox{More stirred, less clumped evolution}{name=stirring,column=2,span=2,above=R200m,bottomaligned=accretion}{
\begin{multicols}{2}
\begin{center}
      \begin{tikzonimage}[width=\linewidth, trim=10 12 5 8, clip]{figures/TmwTntTtot.png}
        %\node at ([text, text width=3cm, align=center] 0.83,.05) {\color{black}\bf\small Avestruz+16};
          \node at ([text, text width=3cm, align=center] 0.37,.93) {\color{blue}\large $-40\%$};
          \node at ([text, text width=3cm, align=center] 0.37,.87) {\color{orange}\large $+45\%$};
          %\draw [draw=black,line width=2pt] (0.45,0.78) rectangle (0.65, 0.84);
          \draw [<->,line width=2pt,black] (0.62,.6) -- (0.62,0.65);
          \node at ([text, text width=3cm, align=center] 0.35,.81) {\color{black}\large $\sim\rm{few}\%$};
          %\node at ([text, text width=3cm, align=center] 0.7,.71) {\color{black}\bf\normalsize };          
          \node at ([text, text width=3cm, align=center] 0.8,.67) {\color{black}\bf\normalsize No evolution};
          \node at ([text, text width=3cm, align=center] 0.88,.625) {\color{black}\bf\normalsize in $\tilde{T}_{tot}$};
          \draw [->,line width=2.5pt,blue] (0.62,.56) -- (0.62,0.48);
          \node at ([text, text width=3cm, align=center] 0.29,.59) {\color{blue}\bf\normalsize Lower $\tilde{T}_{mw}$};
          \node at ([text, text width=3cm, align=center] 0.29,.55) {\color{blue}\bf\normalsize from less };
          \node at ([text, text width=3cm, align=center] 0.36,.5) {\color{blue}\bf\normalsize thermalized gas};
          \draw [->,line width=2.5pt,orange] (0.62,.25) -- (0.62,0.35);
          \node at ([text, text width=3cm, align=center] 0.4,.32) {\color{orange}\bf\normalsize Increasing $\tilde{T}_{nt}$,};
          \node at ([text, text width=3cm, align=center] 0.4,.27) {\color{orange}\bf\normalsize e.g. stirring, with z};
      \end{tikzonimage}
\vspace{-22pt} \captionof{figure}{\large\alignleft Scaled
  ``temperature'' profiles.  The total temperature accounts for gas
  motions/stirring with the non-thermal temperature, $T_{nt}\equiv
  \frac{\mu m_p}{3k_b} \langle v^2_{\text{gas}}\rangle_{\rm mw}$, and
  its evolution scales self-similarly with the critical
  density. \cite{avestruz_etal16} The potential well exhibits the same
  self-similar scaling. \cite{lau_etal15}}
\end{center}
%-------------------------------%
  \begin{center}
    \begin{tikzonimage}[width=\linewidth,trim=0 11 5 6, clip]{figures/TallTbulk.png}
      %\node at ([text, text width=3cm, align=center] 0.83,.05) {\color{black}\bf\small Avestruz+16};
      \draw [->,line width=2pt,red] (0.62,.88) -- (0.62,0.75);
      \draw [->,line width=2.5pt,red] (0.62,.35) -- (0.62,0.26);
      \node at ([text, text width=3cm, align=center] 0.43,.24) {\color{red}\bf\normalsize $\sim10\%$ temperature};
      \node at ([text, text width=3cm, align=center] 0.43,.20) {\color{red}\bf\normalsize suppression};
      \node at ([text, text width=3cm, align=center] 0.49,.16) {\color{red}\bf\normalsize from structures/clumping};
    \end{tikzonimage}
\vspace{-20pt}
\captionof{figure}{\large\alignleft Ratio of average temperature
  profiles with all of the gas to the ``bulk'' component.  At
  ${R}\approx1.5{R}_{500c}$, substructure causes no more than a $10\%$
  evolution in our sample from $0<z<0.7$.  Clumped evolution at this
  radius is comparable for the spectroscopically weighted
  temperature.}
  \end{center}
\end{multicols}
}
%----------------------------------------------------------------------------------------
%	R200m
%----------------------------------------------------------------------------------------
\headerbox{Scale out accretion with $\Delta=200m$}{name=R200m,column=2,span=2,above=bottom,below=stirring,bottomaligned=evolvingaccretion}{
  \begin{multicols}{2}
  \begin{center}
    \begin{tikzonimage}[width=0.93\linewidth, trim=0 0 0 0, clip]{figures/MeanTscaling.png}
        %\node at ([text, text width=3cm, align=center] 0.83,.03) {\color{black}\bf\small Avestruz+16};
        \draw [draw=white,line width=2pt,fill=white] (0.18,0.88) rectangle (0.32, 0.96);
        \node at ([text, text width=3cm, align=center] 0.3,.92) {\color{black}\bf\large $\Delta=200m$};
        \draw [->,line width=2pt,black] (0.58,.78) -- (0.58,0.82);
        \node at ([text, text width=3cm, align=center] 0.62,.88) {\color{black}\bf\large $\sim5\%$};
        \draw [draw=white,line width=2pt,fill=white] (0.18,0.55) rectangle (0.32, 0.63);
        \node at ([text, text width=3cm, align=center] 0.3,.6) {\color{black}\bf\large $\Delta=500m$};
        \draw [->,line width=2pt,black] (0.58,.43) -- (0.58,0.5);
        \node at ([text, text width=3cm, align=center] 0.62,.6) {\color{black}\bf\large $\sim15\%$};
        \draw [draw=white,line width=2pt,fill=white] (0.18,0.25) rectangle (0.32, 0.3);
        \node at ([text, text width=3cm, align=center] 0.3,.27) {\color{black}\bf\large $\Delta=1600m$};
        \draw [->,line width=2pt,black] (0.58,.12) -- (0.58,0.23);
        \node at ([text, text width=3cm, align=center] 0.62,.27) {\color{black}\bf\large $\sim45\%$};
    \end{tikzonimage}
  \end{center}
\vspace{-20pt}
\begin{center}
\captionof{figure}{\large Ratio of temperature profiles, scaled with
  $\Delta_m$, to the $z=0$ profile.  ${R}_{200m}$ scales with the
  average location of the accretion shock \cite{lau_etal15}. Since
  stirring is accretion-driven, $\Delta=200m$ alone scales out the
  temperature evolution.}
\end{center}
\end{multicols}
}

%----------------------------------------------------------------------------------------
%	REFERENCES
%----------------------------------------------------------------------------------------

\headerbox{References}{name=references,column=3,span=1,above=bottom}{
\renewcommand{\section}[2]{\vskip 0.05em} % Get rid of the default "References" section title
\nocite{*} % Insert publications even if they are not cited in the poster
\scriptsize{ % Reduce the font size in this block
\bibliographystyle{unsrt}
\bibliography{sample} % Use sample.bib as the bibliography file
}

}


%----------------------------------------------------------------------------------------
%	RESULTS 2
%----------------------------------------------------------------------------------------

%% \headerbox{Results 2}{name=results2,column=1,below=objectives,bottomaligned=conclusion}{ % This block's bottom aligns with the bottom of the conclusion block

%% Donec faucibus purus at tortor egestas eu fermentum dolor facilisis. Maecenas tempor dui eu neque fringilla rutrum. Mauris \emph{lobortis} nisl accumsan.

%% \begin{center}
%% \begin{tabular}{l l l}
%% \toprule
%% \textbf{Treatments} & \textbf{Response 1} & \textbf{Response 2}\\
%% \midrule
%% Treatment 1 & 0.0003262 & 0.562 \\
%% Treatment 2 & 0.0015681 & 0.910 \\
%% Treatment 3 & 0.0009271 & 0.296 \\
%% \bottomrule
%% \end{tabular}
%% \captionof{table}{Table caption}
%% \end{center}

%% Nulla ut porttitor enim. Suspendisse venenatis dui eget eros gravida tempor. Mauris feugiat elit et augue placerat ultrices. Morbi accumsan enim nec tortor consectetur non commodo.

%% \begin{center}
%% \begin{tabular}{l l l}
%% \toprule
%% \textbf{Treatments} & \textbf{Response 1} & \textbf{Response 2}\\
%% \midrule
%% Treatment 1 & 0.0003262 & 0.562 \\
%% Treatment 2 & 0.0015681 & 0.910 \\
%% Treatment 3 & 0.0009271 & 0.296 \\
%% \bottomrule
%% \end{tabular}
%% \captionof{table}{Table caption}
%% \end{center}
%% }

%----------------------------------------------------------------------------------------

\end{poster}

\end{document}
